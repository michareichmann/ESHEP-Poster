\begin{block}{Conclusion}
	
	\textbf{\underline{Pad Detectors:}}\vspace*{2ex}
	\begin{itemize}
		\item built beam test setup to characterise the rate behaviour of diamond pad detectors
		\item pCVD diamond show non-uniformity according to wide landau of the signal depending on the position in the diamond
		\item rate dependence for most non-irradiated pCVD \SI{<5}{\%}
		\begin{itemize}
			\item unknown origin, maybe surface contamination during production
			\item possible to fix with surface treatment
		\end{itemize}
		\item detectors with irradiated pCVD diamond sensors have a rate dependence below \SI{\sim2}{\%} up to a flux of \SI{20}{\mega\hertz\per \centi\meter^2}
	\end{itemize}\vspace*{2ex}
	
	\textbf{\underline{3D Detectors:}}\vspace*{1ex}
	\begin{itemize}
		\item 3D Detectors work well in pCVD diamond
		\item cell sizes down to \SI{50x50}{\micro\meter}
		\item thin columns down to \SI{2.6}{\micro\meter}
		\item general reasons for inefficiencies:
		\begin{itemize}
			\item less charge in volume of the electrodes (\SI{.4}{\%} for shown devices)
			\item missing/broken columns (\SI{.2}{\%} for the full device)
			\item region with low electric field \ra need precise simulations
		\end{itemize}
		\item \SI{99.2\pm .3}{\%} efficiency in \SI{3x2}{} ganged device
		\item consistent mean charge measurements for all devices: \SI{\sim14500}{e}
		\item largest charge collection of all pCVD diamond detectors
	\end{itemize}

	
\end{block}
